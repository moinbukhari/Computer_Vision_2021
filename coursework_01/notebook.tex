
% Default to the notebook output style

    


% Inherit from the specified cell style.




    
\documentclass[11pt]{article}

    
    
    \usepackage[T1]{fontenc}
    % Nicer default font (+ math font) than Computer Modern for most use cases
    \usepackage{mathpazo}

    % Basic figure setup, for now with no caption control since it's done
    % automatically by Pandoc (which extracts ![](path) syntax from Markdown).
    \usepackage{graphicx}
    % We will generate all images so they have a width \maxwidth. This means
    % that they will get their normal width if they fit onto the page, but
    % are scaled down if they would overflow the margins.
    \makeatletter
    \def\maxwidth{\ifdim\Gin@nat@width>\linewidth\linewidth
    \else\Gin@nat@width\fi}
    \makeatother
    \let\Oldincludegraphics\includegraphics
    % Set max figure width to be 80% of text width, for now hardcoded.
    \renewcommand{\includegraphics}[1]{\Oldincludegraphics[width=.8\maxwidth]{#1}}
    % Ensure that by default, figures have no caption (until we provide a
    % proper Figure object with a Caption API and a way to capture that
    % in the conversion process - todo).
    \usepackage{caption}
    \DeclareCaptionLabelFormat{nolabel}{}
    \captionsetup{labelformat=nolabel}

    \usepackage{adjustbox} % Used to constrain images to a maximum size 
    \usepackage{xcolor} % Allow colors to be defined
    \usepackage{enumerate} % Needed for markdown enumerations to work
    \usepackage{geometry} % Used to adjust the document margins
    \usepackage{amsmath} % Equations
    \usepackage{amssymb} % Equations
    \usepackage{textcomp} % defines textquotesingle
    % Hack from http://tex.stackexchange.com/a/47451/13684:
    \AtBeginDocument{%
        \def\PYZsq{\textquotesingle}% Upright quotes in Pygmentized code
    }
    \usepackage{upquote} % Upright quotes for verbatim code
    \usepackage{eurosym} % defines \euro
    \usepackage[mathletters]{ucs} % Extended unicode (utf-8) support
    \usepackage[utf8x]{inputenc} % Allow utf-8 characters in the tex document
    \usepackage{fancyvrb} % verbatim replacement that allows latex
    \usepackage{grffile} % extends the file name processing of package graphics 
                         % to support a larger range 
    % The hyperref package gives us a pdf with properly built
    % internal navigation ('pdf bookmarks' for the table of contents,
    % internal cross-reference links, web links for URLs, etc.)
    \usepackage{hyperref}
    \usepackage{longtable} % longtable support required by pandoc >1.10
    \usepackage{booktabs}  % table support for pandoc > 1.12.2
    \usepackage[inline]{enumitem} % IRkernel/repr support (it uses the enumerate* environment)
    \usepackage[normalem]{ulem} % ulem is needed to support strikethroughs (\sout)
                                % normalem makes italics be italics, not underlines
    

    
    
    % Colors for the hyperref package
    \definecolor{urlcolor}{rgb}{0,.145,.698}
    \definecolor{linkcolor}{rgb}{.71,0.21,0.01}
    \definecolor{citecolor}{rgb}{.12,.54,.11}

    % ANSI colors
    \definecolor{ansi-black}{HTML}{3E424D}
    \definecolor{ansi-black-intense}{HTML}{282C36}
    \definecolor{ansi-red}{HTML}{E75C58}
    \definecolor{ansi-red-intense}{HTML}{B22B31}
    \definecolor{ansi-green}{HTML}{00A250}
    \definecolor{ansi-green-intense}{HTML}{007427}
    \definecolor{ansi-yellow}{HTML}{DDB62B}
    \definecolor{ansi-yellow-intense}{HTML}{B27D12}
    \definecolor{ansi-blue}{HTML}{208FFB}
    \definecolor{ansi-blue-intense}{HTML}{0065CA}
    \definecolor{ansi-magenta}{HTML}{D160C4}
    \definecolor{ansi-magenta-intense}{HTML}{A03196}
    \definecolor{ansi-cyan}{HTML}{60C6C8}
    \definecolor{ansi-cyan-intense}{HTML}{258F8F}
    \definecolor{ansi-white}{HTML}{C5C1B4}
    \definecolor{ansi-white-intense}{HTML}{A1A6B2}

    % commands and environments needed by pandoc snippets
    % extracted from the output of `pandoc -s`
    \providecommand{\tightlist}{%
      \setlength{\itemsep}{0pt}\setlength{\parskip}{0pt}}
    \DefineVerbatimEnvironment{Highlighting}{Verbatim}{commandchars=\\\{\}}
    % Add ',fontsize=\small' for more characters per line
    \newenvironment{Shaded}{}{}
    \newcommand{\KeywordTok}[1]{\textcolor[rgb]{0.00,0.44,0.13}{\textbf{{#1}}}}
    \newcommand{\DataTypeTok}[1]{\textcolor[rgb]{0.56,0.13,0.00}{{#1}}}
    \newcommand{\DecValTok}[1]{\textcolor[rgb]{0.25,0.63,0.44}{{#1}}}
    \newcommand{\BaseNTok}[1]{\textcolor[rgb]{0.25,0.63,0.44}{{#1}}}
    \newcommand{\FloatTok}[1]{\textcolor[rgb]{0.25,0.63,0.44}{{#1}}}
    \newcommand{\CharTok}[1]{\textcolor[rgb]{0.25,0.44,0.63}{{#1}}}
    \newcommand{\StringTok}[1]{\textcolor[rgb]{0.25,0.44,0.63}{{#1}}}
    \newcommand{\CommentTok}[1]{\textcolor[rgb]{0.38,0.63,0.69}{\textit{{#1}}}}
    \newcommand{\OtherTok}[1]{\textcolor[rgb]{0.00,0.44,0.13}{{#1}}}
    \newcommand{\AlertTok}[1]{\textcolor[rgb]{1.00,0.00,0.00}{\textbf{{#1}}}}
    \newcommand{\FunctionTok}[1]{\textcolor[rgb]{0.02,0.16,0.49}{{#1}}}
    \newcommand{\RegionMarkerTok}[1]{{#1}}
    \newcommand{\ErrorTok}[1]{\textcolor[rgb]{1.00,0.00,0.00}{\textbf{{#1}}}}
    \newcommand{\NormalTok}[1]{{#1}}
    
    % Additional commands for more recent versions of Pandoc
    \newcommand{\ConstantTok}[1]{\textcolor[rgb]{0.53,0.00,0.00}{{#1}}}
    \newcommand{\SpecialCharTok}[1]{\textcolor[rgb]{0.25,0.44,0.63}{{#1}}}
    \newcommand{\VerbatimStringTok}[1]{\textcolor[rgb]{0.25,0.44,0.63}{{#1}}}
    \newcommand{\SpecialStringTok}[1]{\textcolor[rgb]{0.73,0.40,0.53}{{#1}}}
    \newcommand{\ImportTok}[1]{{#1}}
    \newcommand{\DocumentationTok}[1]{\textcolor[rgb]{0.73,0.13,0.13}{\textit{{#1}}}}
    \newcommand{\AnnotationTok}[1]{\textcolor[rgb]{0.38,0.63,0.69}{\textbf{\textit{{#1}}}}}
    \newcommand{\CommentVarTok}[1]{\textcolor[rgb]{0.38,0.63,0.69}{\textbf{\textit{{#1}}}}}
    \newcommand{\VariableTok}[1]{\textcolor[rgb]{0.10,0.09,0.49}{{#1}}}
    \newcommand{\ControlFlowTok}[1]{\textcolor[rgb]{0.00,0.44,0.13}{\textbf{{#1}}}}
    \newcommand{\OperatorTok}[1]{\textcolor[rgb]{0.40,0.40,0.40}{{#1}}}
    \newcommand{\BuiltInTok}[1]{{#1}}
    \newcommand{\ExtensionTok}[1]{{#1}}
    \newcommand{\PreprocessorTok}[1]{\textcolor[rgb]{0.74,0.48,0.00}{{#1}}}
    \newcommand{\AttributeTok}[1]{\textcolor[rgb]{0.49,0.56,0.16}{{#1}}}
    \newcommand{\InformationTok}[1]{\textcolor[rgb]{0.38,0.63,0.69}{\textbf{\textit{{#1}}}}}
    \newcommand{\WarningTok}[1]{\textcolor[rgb]{0.38,0.63,0.69}{\textbf{\textit{{#1}}}}}
    
    
    % Define a nice break command that doesn't care if a line doesn't already
    % exist.
    \def\br{\hspace*{\fill} \\* }
    % Math Jax compatability definitions
    \def\gt{>}
    \def\lt{<}
    % Document parameters
    \title{coursework\_01}
    
    
    

    % Pygments definitions
    
\makeatletter
\def\PY@reset{\let\PY@it=\relax \let\PY@bf=\relax%
    \let\PY@ul=\relax \let\PY@tc=\relax%
    \let\PY@bc=\relax \let\PY@ff=\relax}
\def\PY@tok#1{\csname PY@tok@#1\endcsname}
\def\PY@toks#1+{\ifx\relax#1\empty\else%
    \PY@tok{#1}\expandafter\PY@toks\fi}
\def\PY@do#1{\PY@bc{\PY@tc{\PY@ul{%
    \PY@it{\PY@bf{\PY@ff{#1}}}}}}}
\def\PY#1#2{\PY@reset\PY@toks#1+\relax+\PY@do{#2}}

\expandafter\def\csname PY@tok@w\endcsname{\def\PY@tc##1{\textcolor[rgb]{0.73,0.73,0.73}{##1}}}
\expandafter\def\csname PY@tok@c\endcsname{\let\PY@it=\textit\def\PY@tc##1{\textcolor[rgb]{0.25,0.50,0.50}{##1}}}
\expandafter\def\csname PY@tok@cp\endcsname{\def\PY@tc##1{\textcolor[rgb]{0.74,0.48,0.00}{##1}}}
\expandafter\def\csname PY@tok@k\endcsname{\let\PY@bf=\textbf\def\PY@tc##1{\textcolor[rgb]{0.00,0.50,0.00}{##1}}}
\expandafter\def\csname PY@tok@kp\endcsname{\def\PY@tc##1{\textcolor[rgb]{0.00,0.50,0.00}{##1}}}
\expandafter\def\csname PY@tok@kt\endcsname{\def\PY@tc##1{\textcolor[rgb]{0.69,0.00,0.25}{##1}}}
\expandafter\def\csname PY@tok@o\endcsname{\def\PY@tc##1{\textcolor[rgb]{0.40,0.40,0.40}{##1}}}
\expandafter\def\csname PY@tok@ow\endcsname{\let\PY@bf=\textbf\def\PY@tc##1{\textcolor[rgb]{0.67,0.13,1.00}{##1}}}
\expandafter\def\csname PY@tok@nb\endcsname{\def\PY@tc##1{\textcolor[rgb]{0.00,0.50,0.00}{##1}}}
\expandafter\def\csname PY@tok@nf\endcsname{\def\PY@tc##1{\textcolor[rgb]{0.00,0.00,1.00}{##1}}}
\expandafter\def\csname PY@tok@nc\endcsname{\let\PY@bf=\textbf\def\PY@tc##1{\textcolor[rgb]{0.00,0.00,1.00}{##1}}}
\expandafter\def\csname PY@tok@nn\endcsname{\let\PY@bf=\textbf\def\PY@tc##1{\textcolor[rgb]{0.00,0.00,1.00}{##1}}}
\expandafter\def\csname PY@tok@ne\endcsname{\let\PY@bf=\textbf\def\PY@tc##1{\textcolor[rgb]{0.82,0.25,0.23}{##1}}}
\expandafter\def\csname PY@tok@nv\endcsname{\def\PY@tc##1{\textcolor[rgb]{0.10,0.09,0.49}{##1}}}
\expandafter\def\csname PY@tok@no\endcsname{\def\PY@tc##1{\textcolor[rgb]{0.53,0.00,0.00}{##1}}}
\expandafter\def\csname PY@tok@nl\endcsname{\def\PY@tc##1{\textcolor[rgb]{0.63,0.63,0.00}{##1}}}
\expandafter\def\csname PY@tok@ni\endcsname{\let\PY@bf=\textbf\def\PY@tc##1{\textcolor[rgb]{0.60,0.60,0.60}{##1}}}
\expandafter\def\csname PY@tok@na\endcsname{\def\PY@tc##1{\textcolor[rgb]{0.49,0.56,0.16}{##1}}}
\expandafter\def\csname PY@tok@nt\endcsname{\let\PY@bf=\textbf\def\PY@tc##1{\textcolor[rgb]{0.00,0.50,0.00}{##1}}}
\expandafter\def\csname PY@tok@nd\endcsname{\def\PY@tc##1{\textcolor[rgb]{0.67,0.13,1.00}{##1}}}
\expandafter\def\csname PY@tok@s\endcsname{\def\PY@tc##1{\textcolor[rgb]{0.73,0.13,0.13}{##1}}}
\expandafter\def\csname PY@tok@sd\endcsname{\let\PY@it=\textit\def\PY@tc##1{\textcolor[rgb]{0.73,0.13,0.13}{##1}}}
\expandafter\def\csname PY@tok@si\endcsname{\let\PY@bf=\textbf\def\PY@tc##1{\textcolor[rgb]{0.73,0.40,0.53}{##1}}}
\expandafter\def\csname PY@tok@se\endcsname{\let\PY@bf=\textbf\def\PY@tc##1{\textcolor[rgb]{0.73,0.40,0.13}{##1}}}
\expandafter\def\csname PY@tok@sr\endcsname{\def\PY@tc##1{\textcolor[rgb]{0.73,0.40,0.53}{##1}}}
\expandafter\def\csname PY@tok@ss\endcsname{\def\PY@tc##1{\textcolor[rgb]{0.10,0.09,0.49}{##1}}}
\expandafter\def\csname PY@tok@sx\endcsname{\def\PY@tc##1{\textcolor[rgb]{0.00,0.50,0.00}{##1}}}
\expandafter\def\csname PY@tok@m\endcsname{\def\PY@tc##1{\textcolor[rgb]{0.40,0.40,0.40}{##1}}}
\expandafter\def\csname PY@tok@gh\endcsname{\let\PY@bf=\textbf\def\PY@tc##1{\textcolor[rgb]{0.00,0.00,0.50}{##1}}}
\expandafter\def\csname PY@tok@gu\endcsname{\let\PY@bf=\textbf\def\PY@tc##1{\textcolor[rgb]{0.50,0.00,0.50}{##1}}}
\expandafter\def\csname PY@tok@gd\endcsname{\def\PY@tc##1{\textcolor[rgb]{0.63,0.00,0.00}{##1}}}
\expandafter\def\csname PY@tok@gi\endcsname{\def\PY@tc##1{\textcolor[rgb]{0.00,0.63,0.00}{##1}}}
\expandafter\def\csname PY@tok@gr\endcsname{\def\PY@tc##1{\textcolor[rgb]{1.00,0.00,0.00}{##1}}}
\expandafter\def\csname PY@tok@ge\endcsname{\let\PY@it=\textit}
\expandafter\def\csname PY@tok@gs\endcsname{\let\PY@bf=\textbf}
\expandafter\def\csname PY@tok@gp\endcsname{\let\PY@bf=\textbf\def\PY@tc##1{\textcolor[rgb]{0.00,0.00,0.50}{##1}}}
\expandafter\def\csname PY@tok@go\endcsname{\def\PY@tc##1{\textcolor[rgb]{0.53,0.53,0.53}{##1}}}
\expandafter\def\csname PY@tok@gt\endcsname{\def\PY@tc##1{\textcolor[rgb]{0.00,0.27,0.87}{##1}}}
\expandafter\def\csname PY@tok@err\endcsname{\def\PY@bc##1{\setlength{\fboxsep}{0pt}\fcolorbox[rgb]{1.00,0.00,0.00}{1,1,1}{\strut ##1}}}
\expandafter\def\csname PY@tok@kc\endcsname{\let\PY@bf=\textbf\def\PY@tc##1{\textcolor[rgb]{0.00,0.50,0.00}{##1}}}
\expandafter\def\csname PY@tok@kd\endcsname{\let\PY@bf=\textbf\def\PY@tc##1{\textcolor[rgb]{0.00,0.50,0.00}{##1}}}
\expandafter\def\csname PY@tok@kn\endcsname{\let\PY@bf=\textbf\def\PY@tc##1{\textcolor[rgb]{0.00,0.50,0.00}{##1}}}
\expandafter\def\csname PY@tok@kr\endcsname{\let\PY@bf=\textbf\def\PY@tc##1{\textcolor[rgb]{0.00,0.50,0.00}{##1}}}
\expandafter\def\csname PY@tok@bp\endcsname{\def\PY@tc##1{\textcolor[rgb]{0.00,0.50,0.00}{##1}}}
\expandafter\def\csname PY@tok@fm\endcsname{\def\PY@tc##1{\textcolor[rgb]{0.00,0.00,1.00}{##1}}}
\expandafter\def\csname PY@tok@vc\endcsname{\def\PY@tc##1{\textcolor[rgb]{0.10,0.09,0.49}{##1}}}
\expandafter\def\csname PY@tok@vg\endcsname{\def\PY@tc##1{\textcolor[rgb]{0.10,0.09,0.49}{##1}}}
\expandafter\def\csname PY@tok@vi\endcsname{\def\PY@tc##1{\textcolor[rgb]{0.10,0.09,0.49}{##1}}}
\expandafter\def\csname PY@tok@vm\endcsname{\def\PY@tc##1{\textcolor[rgb]{0.10,0.09,0.49}{##1}}}
\expandafter\def\csname PY@tok@sa\endcsname{\def\PY@tc##1{\textcolor[rgb]{0.73,0.13,0.13}{##1}}}
\expandafter\def\csname PY@tok@sb\endcsname{\def\PY@tc##1{\textcolor[rgb]{0.73,0.13,0.13}{##1}}}
\expandafter\def\csname PY@tok@sc\endcsname{\def\PY@tc##1{\textcolor[rgb]{0.73,0.13,0.13}{##1}}}
\expandafter\def\csname PY@tok@dl\endcsname{\def\PY@tc##1{\textcolor[rgb]{0.73,0.13,0.13}{##1}}}
\expandafter\def\csname PY@tok@s2\endcsname{\def\PY@tc##1{\textcolor[rgb]{0.73,0.13,0.13}{##1}}}
\expandafter\def\csname PY@tok@sh\endcsname{\def\PY@tc##1{\textcolor[rgb]{0.73,0.13,0.13}{##1}}}
\expandafter\def\csname PY@tok@s1\endcsname{\def\PY@tc##1{\textcolor[rgb]{0.73,0.13,0.13}{##1}}}
\expandafter\def\csname PY@tok@mb\endcsname{\def\PY@tc##1{\textcolor[rgb]{0.40,0.40,0.40}{##1}}}
\expandafter\def\csname PY@tok@mf\endcsname{\def\PY@tc##1{\textcolor[rgb]{0.40,0.40,0.40}{##1}}}
\expandafter\def\csname PY@tok@mh\endcsname{\def\PY@tc##1{\textcolor[rgb]{0.40,0.40,0.40}{##1}}}
\expandafter\def\csname PY@tok@mi\endcsname{\def\PY@tc##1{\textcolor[rgb]{0.40,0.40,0.40}{##1}}}
\expandafter\def\csname PY@tok@il\endcsname{\def\PY@tc##1{\textcolor[rgb]{0.40,0.40,0.40}{##1}}}
\expandafter\def\csname PY@tok@mo\endcsname{\def\PY@tc##1{\textcolor[rgb]{0.40,0.40,0.40}{##1}}}
\expandafter\def\csname PY@tok@ch\endcsname{\let\PY@it=\textit\def\PY@tc##1{\textcolor[rgb]{0.25,0.50,0.50}{##1}}}
\expandafter\def\csname PY@tok@cm\endcsname{\let\PY@it=\textit\def\PY@tc##1{\textcolor[rgb]{0.25,0.50,0.50}{##1}}}
\expandafter\def\csname PY@tok@cpf\endcsname{\let\PY@it=\textit\def\PY@tc##1{\textcolor[rgb]{0.25,0.50,0.50}{##1}}}
\expandafter\def\csname PY@tok@c1\endcsname{\let\PY@it=\textit\def\PY@tc##1{\textcolor[rgb]{0.25,0.50,0.50}{##1}}}
\expandafter\def\csname PY@tok@cs\endcsname{\let\PY@it=\textit\def\PY@tc##1{\textcolor[rgb]{0.25,0.50,0.50}{##1}}}

\def\PYZbs{\char`\\}
\def\PYZus{\char`\_}
\def\PYZob{\char`\{}
\def\PYZcb{\char`\}}
\def\PYZca{\char`\^}
\def\PYZam{\char`\&}
\def\PYZlt{\char`\<}
\def\PYZgt{\char`\>}
\def\PYZsh{\char`\#}
\def\PYZpc{\char`\%}
\def\PYZdl{\char`\$}
\def\PYZhy{\char`\-}
\def\PYZsq{\char`\'}
\def\PYZdq{\char`\"}
\def\PYZti{\char`\~}
% for compatibility with earlier versions
\def\PYZat{@}
\def\PYZlb{[}
\def\PYZrb{]}
\makeatother


    % Exact colors from NB
    \definecolor{incolor}{rgb}{0.0, 0.0, 0.5}
    \definecolor{outcolor}{rgb}{0.545, 0.0, 0.0}



    
    % Prevent overflowing lines due to hard-to-break entities
    \sloppy 
    % Setup hyperref package
    \hypersetup{
      breaklinks=true,  % so long urls are correctly broken across lines
      colorlinks=true,
      urlcolor=urlcolor,
      linkcolor=linkcolor,
      citecolor=citecolor,
      }
    % Slightly bigger margins than the latex defaults
    
    \geometry{verbose,tmargin=1in,bmargin=1in,lmargin=1in,rmargin=1in}
    
    

    \begin{document}
    
    
    \maketitle
    
    

    
    \section{Coursework 1: Image
filtering}\label{coursework-1-image-filtering}

In this coursework you will practice image filtering techniques, which
are commonly used to smooth, sharpen or add certain effects to images.
The coursework includes both coding questions and written questions.
Please read both the text and code comment in this notebook to get an
idea what you are expected to implement.

\subsection{What to do?}\label{what-to-do}

\begin{itemize}
\item
  Complete and run the code using \texttt{jupyter-lab} or
  \texttt{jupyter-notebook} to get the results.
\item
  Export (File \textbar{} Export Notebook As...) or print (using the
  print function of your browser) the notebook as a pdf file, which
  contains your code, results and text answers, and upload the pdf file
  onto \href{https://cate.doc.ic.ac.uk}{Cate}.
\end{itemize}

\subsection{Dependencies:}\label{dependencies}

If you do not have Jupyter-Lab on your laptop, you can find information
for installing Jupyter-Lab
\href{https://jupyterlab.readthedocs.io/en/stable/getting_started/installation.html}{here}.

There may be certain Python packages you may want to use for completing
the coursework. We have provided examples below for importing libraries.
If some packages are missing, you need to install them. In general, new
packages (e.g. imageio etc) can be installed by running

\texttt{pip3\ install\ {[}package\_name{]}}

in the terminal. If you use Anaconda, you can also install new packages
by running \texttt{conda\ install\ {[}package\_name{]}} or using its
graphic user interface.

    \begin{Verbatim}[commandchars=\\\{\}]
{\color{incolor}In [{\color{incolor}1}]:} \PY{c+c1}{\PYZsh{} Import libaries (provided)}
        \PY{k+kn}{import} \PY{n+nn}{imageio}
        \PY{k+kn}{import} \PY{n+nn}{numpy} \PY{k}{as} \PY{n+nn}{np}
        \PY{k+kn}{import} \PY{n+nn}{matplotlib}\PY{n+nn}{.}\PY{n+nn}{pyplot} \PY{k}{as} \PY{n+nn}{plt}
        \PY{k+kn}{import} \PY{n+nn}{noise}
        \PY{k+kn}{import} \PY{n+nn}{scipy}
        \PY{k+kn}{import} \PY{n+nn}{scipy}\PY{n+nn}{.}\PY{n+nn}{signal}
        \PY{k+kn}{import} \PY{n+nn}{math}
        \PY{k+kn}{import} \PY{n+nn}{time}
\end{Verbatim}


    \subsection{1. Moving average filter (20
points).}\label{moving-average-filter-20-points.}

Read a specific input image and add noise to the image. Design a moving
average filter of kernel size 3x3 and 11x11 respectively. Perform image
filtering on the noisy image.

Design the kernel of the filter by yourself. Then perform 2D image
filtering using the function \texttt{scipy.signal.convolve2d()}.

    \begin{Verbatim}[commandchars=\\\{\}]
{\color{incolor}In [{\color{incolor}2}]:} \PY{c+c1}{\PYZsh{} Read the image (provided)}
        \PY{n}{image} \PY{o}{=} \PY{n}{imageio}\PY{o}{.}\PY{n}{imread}\PY{p}{(}\PY{l+s+s1}{\PYZsq{}}\PY{l+s+s1}{london.jpg}\PY{l+s+s1}{\PYZsq{}}\PY{p}{)}
        \PY{n}{plt}\PY{o}{.}\PY{n}{imshow}\PY{p}{(}\PY{n}{image}\PY{p}{,} \PY{n}{cmap}\PY{o}{=}\PY{l+s+s1}{\PYZsq{}}\PY{l+s+s1}{gray}\PY{l+s+s1}{\PYZsq{}}\PY{p}{)}
        \PY{n}{plt}\PY{o}{.}\PY{n}{gcf}\PY{p}{(}\PY{p}{)}\PY{o}{.}\PY{n}{set\PYZus{}size\PYZus{}inches}\PY{p}{(}\PY{l+m+mi}{10}\PY{p}{,} \PY{l+m+mi}{8}\PY{p}{)}
\end{Verbatim}


    \begin{center}
    \adjustimage{max size={0.9\linewidth}{0.9\paperheight}}{output_3_0.png}
    \end{center}
    { \hspace*{\fill} \\}
    
    \begin{Verbatim}[commandchars=\\\{\}]
{\color{incolor}In [{\color{incolor}3}]:} \PY{c+c1}{\PYZsh{} Corrupt the image with Gaussian noise (provided)}
        \PY{n}{image\PYZus{}noisy} \PY{o}{=} \PY{n}{noise}\PY{o}{.}\PY{n}{add\PYZus{}noise}\PY{p}{(}\PY{n}{image}\PY{p}{,} \PY{l+s+s1}{\PYZsq{}}\PY{l+s+s1}{gaussian}\PY{l+s+s1}{\PYZsq{}}\PY{p}{)}
        \PY{n}{plt}\PY{o}{.}\PY{n}{imshow}\PY{p}{(}\PY{n}{image\PYZus{}noisy}\PY{p}{,} \PY{n}{cmap}\PY{o}{=}\PY{l+s+s1}{\PYZsq{}}\PY{l+s+s1}{gray}\PY{l+s+s1}{\PYZsq{}}\PY{p}{)}
        \PY{n}{plt}\PY{o}{.}\PY{n}{gcf}\PY{p}{(}\PY{p}{)}\PY{o}{.}\PY{n}{set\PYZus{}size\PYZus{}inches}\PY{p}{(}\PY{l+m+mi}{10}\PY{p}{,} \PY{l+m+mi}{8}\PY{p}{)}
\end{Verbatim}


    \begin{center}
    \adjustimage{max size={0.9\linewidth}{0.9\paperheight}}{output_4_0.png}
    \end{center}
    { \hspace*{\fill} \\}
    
    \subsubsection{Note: from now on, please use the noisy image as the
input for the
filters.}\label{note-from-now-on-please-use-the-noisy-image-as-the-input-for-the-filters.}

\subsubsection{1.1 Filter the noisy image with a 3x3 moving average
filter. Show the filtering results. (5
points)}\label{filter-the-noisy-image-with-a-3x3-moving-average-filter.-show-the-filtering-results.-5-points}

    \begin{Verbatim}[commandchars=\\\{\}]
{\color{incolor}In [{\color{incolor}4}]:} \PY{c+c1}{\PYZsh{} Design the filter h}
        \PY{n}{rows}\PY{p}{,} \PY{n}{cols} \PY{o}{=} \PY{p}{(}\PY{l+m+mi}{3}\PY{p}{,} \PY{l+m+mi}{3}\PY{p}{)} 
        \PY{n}{h} \PY{o}{=} \PY{p}{(}\PY{p}{[}\PY{p}{[}\PY{l+m+mi}{1}\PY{o}{/}\PY{p}{(}\PY{n}{cols}\PY{o}{*}\PY{n}{rows}\PY{p}{)}\PY{p}{]}\PY{o}{*}\PY{n}{cols}\PY{p}{]}\PY{o}{*}\PY{n}{rows}\PY{p}{)}
        
        \PY{c+c1}{\PYZsh{} Convolve the corrupted image with h using scipy.signal.convolve2d function}
        \PY{n}{image\PYZus{}filtered} \PY{o}{=} \PY{n}{scipy}\PY{o}{.}\PY{n}{signal}\PY{o}{.}\PY{n}{convolve2d}\PY{p}{(}\PY{n}{image\PYZus{}noisy}\PY{p}{,}\PY{n}{h}\PY{p}{)}
        
        \PY{c+c1}{\PYZsh{} Print the filter (provided)}
        \PY{n+nb}{print}\PY{p}{(}\PY{l+s+s1}{\PYZsq{}}\PY{l+s+s1}{Filter h:}\PY{l+s+s1}{\PYZsq{}}\PY{p}{)}
        \PY{n+nb}{print}\PY{p}{(}\PY{n}{h}\PY{p}{)}
        
        \PY{c+c1}{\PYZsh{} Display the filtering result (provided)}
        \PY{n}{plt}\PY{o}{.}\PY{n}{imshow}\PY{p}{(}\PY{n}{image\PYZus{}filtered}\PY{p}{,} \PY{n}{cmap}\PY{o}{=}\PY{l+s+s1}{\PYZsq{}}\PY{l+s+s1}{gray}\PY{l+s+s1}{\PYZsq{}}\PY{p}{)}
        \PY{n}{plt}\PY{o}{.}\PY{n}{gcf}\PY{p}{(}\PY{p}{)}\PY{o}{.}\PY{n}{set\PYZus{}size\PYZus{}inches}\PY{p}{(}\PY{l+m+mi}{10}\PY{p}{,} \PY{l+m+mi}{8}\PY{p}{)}
\end{Verbatim}


    \begin{Verbatim}[commandchars=\\\{\}]
Filter h:
[[0.1111111111111111, 0.1111111111111111, 0.1111111111111111], [0.1111111111111111, 0.1111111111111111, 0.1111111111111111], [0.1111111111111111, 0.1111111111111111, 0.1111111111111111]]

    \end{Verbatim}

    \begin{center}
    \adjustimage{max size={0.9\linewidth}{0.9\paperheight}}{output_6_1.png}
    \end{center}
    { \hspace*{\fill} \\}
    
    \subsubsection{1.2 Filter the noisy image with a 11x11 moving average
filter. (5
points)}\label{filter-the-noisy-image-with-a-11x11-moving-average-filter.-5-points}

    \begin{Verbatim}[commandchars=\\\{\}]
{\color{incolor}In [{\color{incolor}5}]:} \PY{c+c1}{\PYZsh{} Design the filter h}
        \PY{n}{rows}\PY{p}{,} \PY{n}{cols} \PY{o}{=} \PY{p}{(}\PY{l+m+mi}{11}\PY{p}{,} \PY{l+m+mi}{11}\PY{p}{)} 
        \PY{n}{h} \PY{o}{=} \PY{p}{(}\PY{p}{[}\PY{p}{[}\PY{l+m+mi}{1}\PY{o}{/}\PY{p}{(}\PY{n}{cols}\PY{o}{*}\PY{n}{rows}\PY{p}{)}\PY{p}{]}\PY{o}{*}\PY{n}{cols}\PY{p}{]}\PY{o}{*}\PY{n}{rows}\PY{p}{)}
        
        \PY{c+c1}{\PYZsh{} Convolve the corrupted image with h using scipy.signal.convolve2d function}
        \PY{n}{image\PYZus{}filtered} \PY{o}{=} \PY{n}{scipy}\PY{o}{.}\PY{n}{signal}\PY{o}{.}\PY{n}{convolve2d}\PY{p}{(}\PY{n}{image\PYZus{}noisy}\PY{p}{,}\PY{n}{h}\PY{p}{)}
        
        \PY{c+c1}{\PYZsh{} Print the filter (provided)}
        \PY{n+nb}{print}\PY{p}{(}\PY{l+s+s1}{\PYZsq{}}\PY{l+s+s1}{Filter h:}\PY{l+s+s1}{\PYZsq{}}\PY{p}{)}
        \PY{n+nb}{print}\PY{p}{(}\PY{n}{h}\PY{p}{)}
        
        \PY{c+c1}{\PYZsh{} Display the filtering result (provided)}
        \PY{n}{plt}\PY{o}{.}\PY{n}{imshow}\PY{p}{(}\PY{n}{image\PYZus{}filtered}\PY{p}{,} \PY{n}{cmap}\PY{o}{=}\PY{l+s+s1}{\PYZsq{}}\PY{l+s+s1}{gray}\PY{l+s+s1}{\PYZsq{}}\PY{p}{)}
        \PY{n}{plt}\PY{o}{.}\PY{n}{gcf}\PY{p}{(}\PY{p}{)}\PY{o}{.}\PY{n}{set\PYZus{}size\PYZus{}inches}\PY{p}{(}\PY{l+m+mi}{10}\PY{p}{,} \PY{l+m+mi}{8}\PY{p}{)}
\end{Verbatim}


    \begin{Verbatim}[commandchars=\\\{\}]
Filter h:
[[0.008264462809917356, 0.008264462809917356, 0.008264462809917356, 0.008264462809917356, 0.008264462809917356, 0.008264462809917356, 0.008264462809917356, 0.008264462809917356, 0.008264462809917356, 0.008264462809917356, 0.008264462809917356], [0.008264462809917356, 0.008264462809917356, 0.008264462809917356, 0.008264462809917356, 0.008264462809917356, 0.008264462809917356, 0.008264462809917356, 0.008264462809917356, 0.008264462809917356, 0.008264462809917356, 0.008264462809917356], [0.008264462809917356, 0.008264462809917356, 0.008264462809917356, 0.008264462809917356, 0.008264462809917356, 0.008264462809917356, 0.008264462809917356, 0.008264462809917356, 0.008264462809917356, 0.008264462809917356, 0.008264462809917356], [0.008264462809917356, 0.008264462809917356, 0.008264462809917356, 0.008264462809917356, 0.008264462809917356, 0.008264462809917356, 0.008264462809917356, 0.008264462809917356, 0.008264462809917356, 0.008264462809917356, 0.008264462809917356], [0.008264462809917356, 0.008264462809917356, 0.008264462809917356, 0.008264462809917356, 0.008264462809917356, 0.008264462809917356, 0.008264462809917356, 0.008264462809917356, 0.008264462809917356, 0.008264462809917356, 0.008264462809917356], [0.008264462809917356, 0.008264462809917356, 0.008264462809917356, 0.008264462809917356, 0.008264462809917356, 0.008264462809917356, 0.008264462809917356, 0.008264462809917356, 0.008264462809917356, 0.008264462809917356, 0.008264462809917356], [0.008264462809917356, 0.008264462809917356, 0.008264462809917356, 0.008264462809917356, 0.008264462809917356, 0.008264462809917356, 0.008264462809917356, 0.008264462809917356, 0.008264462809917356, 0.008264462809917356, 0.008264462809917356], [0.008264462809917356, 0.008264462809917356, 0.008264462809917356, 0.008264462809917356, 0.008264462809917356, 0.008264462809917356, 0.008264462809917356, 0.008264462809917356, 0.008264462809917356, 0.008264462809917356, 0.008264462809917356], [0.008264462809917356, 0.008264462809917356, 0.008264462809917356, 0.008264462809917356, 0.008264462809917356, 0.008264462809917356, 0.008264462809917356, 0.008264462809917356, 0.008264462809917356, 0.008264462809917356, 0.008264462809917356], [0.008264462809917356, 0.008264462809917356, 0.008264462809917356, 0.008264462809917356, 0.008264462809917356, 0.008264462809917356, 0.008264462809917356, 0.008264462809917356, 0.008264462809917356, 0.008264462809917356, 0.008264462809917356], [0.008264462809917356, 0.008264462809917356, 0.008264462809917356, 0.008264462809917356, 0.008264462809917356, 0.008264462809917356, 0.008264462809917356, 0.008264462809917356, 0.008264462809917356, 0.008264462809917356, 0.008264462809917356]]

    \end{Verbatim}

    \begin{center}
    \adjustimage{max size={0.9\linewidth}{0.9\paperheight}}{output_8_1.png}
    \end{center}
    { \hspace*{\fill} \\}
    
    \subsubsection{1.3 Comment on the filtering results. How do different
kernel sizes influence the filtering results? (10
points)}\label{comment-on-the-filtering-results.-how-do-different-kernel-sizes-influence-the-filtering-results-10-points}

    The 11x11 moving average filter had a greater smoothing effect than the
3x3 kernel. The greater thekernel size the greater the smoothing effect.
A very large kernel could lead to blurring. This is because larger
kernel size takes into account a greater area of the image when
calculating average which leads to get a greater blurring effect.

    \subsection{2. Edge detection (65
points).}\label{edge-detection-65-points.}

Perform edge detection using Sobel filters, as well as Gaussian + Sobel
filters for edge detection.

    \subsubsection{2.1 Implement 3x3 Sobel filters and convolve with the
noisy image. (10
points)}\label{implement-3x3-sobel-filters-and-convolve-with-the-noisy-image.-10-points}

    \begin{Verbatim}[commandchars=\\\{\}]
{\color{incolor}In [{\color{incolor}6}]:} \PY{c+c1}{\PYZsh{} Design the Sobel filters}
        \PY{n}{h\PYZus{}sobel\PYZus{}x} \PY{o}{=} \PY{p}{[}\PY{p}{[}\PY{l+m+mi}{1}\PY{p}{,}\PY{l+m+mi}{0}\PY{p}{,}\PY{o}{\PYZhy{}}\PY{l+m+mi}{1}\PY{p}{]}\PY{p}{,}\PY{p}{[}\PY{l+m+mi}{2}\PY{p}{,}\PY{l+m+mi}{0}\PY{p}{,}\PY{o}{\PYZhy{}}\PY{l+m+mi}{2}\PY{p}{]}\PY{p}{,}\PY{p}{[}\PY{l+m+mi}{1}\PY{p}{,}\PY{l+m+mi}{0}\PY{p}{,}\PY{o}{\PYZhy{}}\PY{l+m+mi}{1}\PY{p}{]}\PY{p}{]}
        \PY{n}{h\PYZus{}sobel\PYZus{}y} \PY{o}{=} \PY{p}{[}\PY{p}{[}\PY{l+m+mi}{1}\PY{p}{,}\PY{l+m+mi}{2}\PY{p}{,}\PY{l+m+mi}{1}\PY{p}{]}\PY{p}{,}\PY{p}{[}\PY{l+m+mi}{0}\PY{p}{,}\PY{l+m+mi}{0}\PY{p}{,}\PY{l+m+mi}{0}\PY{p}{]}\PY{p}{,}\PY{p}{[}\PY{o}{\PYZhy{}}\PY{l+m+mi}{1}\PY{p}{,}\PY{o}{\PYZhy{}}\PY{l+m+mi}{2}\PY{p}{,}\PY{o}{\PYZhy{}}\PY{l+m+mi}{1}\PY{p}{]}\PY{p}{]}
        
        \PY{c+c1}{\PYZsh{} Sobel filtering}
        \PY{n}{g\PYZus{}x} \PY{o}{=} \PY{n}{scipy}\PY{o}{.}\PY{n}{signal}\PY{o}{.}\PY{n}{convolve2d}\PY{p}{(}\PY{n}{image\PYZus{}noisy}\PY{p}{,}\PY{n}{h\PYZus{}sobel\PYZus{}x}\PY{p}{)}
        \PY{n}{g\PYZus{}y} \PY{o}{=} \PY{n}{scipy}\PY{o}{.}\PY{n}{signal}\PY{o}{.}\PY{n}{convolve2d}\PY{p}{(}\PY{n}{image\PYZus{}noisy}\PY{p}{,}\PY{n}{h\PYZus{}sobel\PYZus{}y}\PY{p}{)}
        
        \PY{c+c1}{\PYZsh{} Calculate the gradient magnitude}
        \PY{n}{g\PYZus{}squared} \PY{o}{=} \PY{n}{np}\PY{o}{.}\PY{n}{add}\PY{p}{(}\PY{n}{np}\PY{o}{.}\PY{n}{square}\PY{p}{(}\PY{n}{g\PYZus{}x}\PY{p}{)}\PY{p}{,}\PY{n}{np}\PY{o}{.}\PY{n}{square}\PY{p}{(}\PY{n}{g\PYZus{}y}\PY{p}{)}\PY{p}{)}
        \PY{n}{sobel\PYZus{}mag} \PY{o}{=} \PY{n}{np}\PY{o}{.}\PY{n}{sqrt}\PY{p}{(}\PY{n}{g\PYZus{}squared}\PY{p}{)}
        
        \PY{c+c1}{\PYZsh{} Print the filters (provided)}
        \PY{n+nb}{print}\PY{p}{(}\PY{l+s+s1}{\PYZsq{}}\PY{l+s+s1}{h\PYZus{}sobel\PYZus{}x:}\PY{l+s+s1}{\PYZsq{}}\PY{p}{)}
        \PY{n+nb}{print}\PY{p}{(}\PY{n}{h\PYZus{}sobel\PYZus{}x}\PY{p}{)}
        \PY{n+nb}{print}\PY{p}{(}\PY{l+s+s1}{\PYZsq{}}\PY{l+s+s1}{h\PYZus{}sobel\PYZus{}y:}\PY{l+s+s1}{\PYZsq{}}\PY{p}{)}
        \PY{n+nb}{print}\PY{p}{(}\PY{n}{h\PYZus{}sobel\PYZus{}y}\PY{p}{)}
        
        \PY{c+c1}{\PYZsh{} Display the magnitude image (provided)}
        \PY{n}{plt}\PY{o}{.}\PY{n}{imshow}\PY{p}{(}\PY{n}{sobel\PYZus{}mag}\PY{p}{,} \PY{n}{cmap}\PY{o}{=}\PY{l+s+s1}{\PYZsq{}}\PY{l+s+s1}{gray}\PY{l+s+s1}{\PYZsq{}}\PY{p}{)}
        \PY{n}{plt}\PY{o}{.}\PY{n}{gcf}\PY{p}{(}\PY{p}{)}\PY{o}{.}\PY{n}{set\PYZus{}size\PYZus{}inches}\PY{p}{(}\PY{l+m+mi}{10}\PY{p}{,} \PY{l+m+mi}{8}\PY{p}{)}
\end{Verbatim}


    \begin{Verbatim}[commandchars=\\\{\}]
h\_sobel\_x:
[[1, 0, -1], [2, 0, -2], [1, 0, -1]]
h\_sobel\_y:
[[1, 2, 1], [0, 0, 0], [-1, -2, -1]]

    \end{Verbatim}

    \begin{center}
    \adjustimage{max size={0.9\linewidth}{0.9\paperheight}}{output_13_1.png}
    \end{center}
    { \hspace*{\fill} \\}
    
    \subsubsection{\texorpdfstring{2.2 Implement a function that generates a
2D Gaussian filter given the parameter \(\sigma\). (10
points)}{2.2 Implement a function that generates a 2D Gaussian filter given the parameter \textbackslash{}sigma. (10 points)}}\label{implement-a-function-that-generates-a-2d-gaussian-filter-given-the-parameter-sigma.-10-points}

    \begin{Verbatim}[commandchars=\\\{\}]
{\color{incolor}In [{\color{incolor}7}]:} \PY{c+c1}{\PYZsh{} Design the Gaussian filter}
        \PY{k}{def} \PY{n+nf}{gaussian\PYZus{}filter\PYZus{}2d}\PY{p}{(}\PY{n}{sigma}\PY{p}{)}\PY{p}{:}
            \PY{c+c1}{\PYZsh{} sigma: the parameter sigma in the Gaussian kernel (unit: pixel)}
            \PY{c+c1}{\PYZsh{}}
            \PY{c+c1}{\PYZsh{} return: a 2D array for the Gaussian kernel}
            \PY{n}{constant} \PY{o}{=} \PY{l+m+mi}{1}\PY{o}{/}\PY{p}{(}\PY{l+m+mi}{2}\PY{o}{*}\PY{n}{np}\PY{o}{.}\PY{n}{pi}\PY{o}{*}\PY{p}{(}\PY{n}{sigma}\PY{o}{*}\PY{o}{*}\PY{l+m+mi}{2}\PY{p}{)}\PY{p}{)}
            \PY{n}{k} \PY{o}{=} \PY{l+m+mi}{4}
            \PY{n}{size} \PY{o}{=} \PY{n+nb}{int}\PY{p}{(}\PY{l+m+mi}{2}\PY{o}{*}\PY{n}{sigma} \PY{o}{*} \PY{n}{k}\PY{p}{)}\PY{o}{+}\PY{l+m+mi}{1} \PY{c+c1}{\PYZsh{}kernel size}
            \PY{n}{var} \PY{o}{=} \PY{n}{sigma}\PY{o}{*}\PY{o}{*}\PY{l+m+mi}{2} \PY{c+c1}{\PYZsh{}variance}
            \PY{n}{h} \PY{o}{=} \PY{p}{[}\PY{p}{]}  \PY{c+c1}{\PYZsh{}initialise 2d kernel}
            
            \PY{k}{for} \PY{n}{i} \PY{o+ow}{in} \PY{n+nb}{range}\PY{p}{(}\PY{n}{size}\PY{p}{)}\PY{p}{:}
                \PY{n}{h}\PY{o}{.}\PY{n}{append}\PY{p}{(}\PY{p}{[}\PY{p}{]}\PY{p}{)}
                \PY{n}{x} \PY{o}{=} \PY{n}{i}\PY{o}{\PYZhy{}}\PY{p}{(}\PY{n}{k}\PY{o}{*}\PY{n}{sigma}\PY{p}{)}
                \PY{k}{for} \PY{n}{j} \PY{o+ow}{in} \PY{n+nb}{range}\PY{p}{(}\PY{n}{size}\PY{p}{)}\PY{p}{:}
                    \PY{n}{y} \PY{o}{=} \PY{n}{j}\PY{o}{\PYZhy{}}\PY{p}{(}\PY{n}{k}\PY{o}{*}\PY{n}{sigma}\PY{p}{)}
                    \PY{c+c1}{\PYZsh{}print(x,\PYZdq{},\PYZdq{},y)}
                    \PY{c+c1}{\PYZsh{}print(constant*np.exp(\PYZhy{}(x**2 + y**2)/(var*2)))}
                    \PY{n}{h}\PY{p}{[}\PY{n}{i}\PY{p}{]}\PY{o}{.}\PY{n}{append}\PY{p}{(}\PY{n}{constant}\PY{o}{*}\PY{n}{np}\PY{o}{.}\PY{n}{exp}\PY{p}{(}\PY{o}{\PYZhy{}}\PY{p}{(}\PY{n}{x}\PY{o}{*}\PY{o}{*}\PY{l+m+mi}{2} \PY{o}{+} \PY{n}{y}\PY{o}{*}\PY{o}{*}\PY{l+m+mi}{2}\PY{p}{)}\PY{o}{/}\PY{p}{(}\PY{n}{var}\PY{o}{*}\PY{l+m+mi}{2}\PY{p}{)}\PY{p}{)}\PY{p}{)}
                    \PY{c+c1}{\PYZsh{}print(h)}
            \PY{k}{return} \PY{n}{h}
        
        \PY{c+c1}{\PYZsh{} Visualise the Gaussian filter when sigma = 3 pixel (provided)}
        \PY{n}{sigma} \PY{o}{=} \PY{l+m+mi}{3}
        \PY{n}{h} \PY{o}{=} \PY{n}{gaussian\PYZus{}filter\PYZus{}2d}\PY{p}{(}\PY{n}{sigma}\PY{p}{)}
        \PY{n}{plt}\PY{o}{.}\PY{n}{imshow}\PY{p}{(}\PY{n}{h}\PY{p}{)}
\end{Verbatim}


\begin{Verbatim}[commandchars=\\\{\}]
{\color{outcolor}Out[{\color{outcolor}7}]:} <matplotlib.image.AxesImage at 0x7f0b4c462080>
\end{Verbatim}
            
    \begin{center}
    \adjustimage{max size={0.9\linewidth}{0.9\paperheight}}{output_15_1.png}
    \end{center}
    { \hspace*{\fill} \\}
    
    \subsubsection{\texorpdfstring{2.3 Perform Gaussian smoothing
(\(\sigma\) = 3 pixels), followed by Sobel filtering, show the gradient
magnitude image. (7
points)}{2.3 Perform Gaussian smoothing (\textbackslash{}sigma = 3 pixels), followed by Sobel filtering, show the gradient magnitude image. (7 points)}}\label{perform-gaussian-smoothing-sigma-3-pixels-followed-by-sobel-filtering-show-the-gradient-magnitude-image.-7-points}

    \begin{Verbatim}[commandchars=\\\{\}]
{\color{incolor}In [{\color{incolor}8}]:} \PY{c+c1}{\PYZsh{} Perform Gaussian smoothing before Sobel filtering}
        \PY{n}{gauss\PYZus{}filter} \PY{o}{=} \PY{n}{gaussian\PYZus{}filter\PYZus{}2d}\PY{p}{(}\PY{l+m+mi}{3}\PY{p}{)}
        \PY{n}{gauss\PYZus{}img} \PY{o}{=} \PY{n}{scipy}\PY{o}{.}\PY{n}{signal}\PY{o}{.}\PY{n}{convolve2d}\PY{p}{(}\PY{n}{image\PYZus{}noisy}\PY{p}{,} \PY{n}{gauss\PYZus{}filter}\PY{p}{)}
        
        \PY{c+c1}{\PYZsh{} \PYZsh{} Sobel filtering}
        \PY{n}{h\PYZus{}sobel\PYZus{}x} \PY{o}{=} \PY{p}{[}\PY{p}{[}\PY{l+m+mi}{1}\PY{p}{,}\PY{l+m+mi}{0}\PY{p}{,}\PY{o}{\PYZhy{}}\PY{l+m+mi}{1}\PY{p}{]}\PY{p}{,}\PY{p}{[}\PY{l+m+mi}{2}\PY{p}{,}\PY{l+m+mi}{0}\PY{p}{,}\PY{o}{\PYZhy{}}\PY{l+m+mi}{2}\PY{p}{]}\PY{p}{,}\PY{p}{[}\PY{l+m+mi}{1}\PY{p}{,}\PY{l+m+mi}{0}\PY{p}{,}\PY{o}{\PYZhy{}}\PY{l+m+mi}{1}\PY{p}{]}\PY{p}{]}
        \PY{n}{h\PYZus{}sobel\PYZus{}y} \PY{o}{=} \PY{p}{[}\PY{p}{[}\PY{l+m+mi}{1}\PY{p}{,}\PY{l+m+mi}{2}\PY{p}{,}\PY{l+m+mi}{1}\PY{p}{]}\PY{p}{,}\PY{p}{[}\PY{l+m+mi}{0}\PY{p}{,}\PY{l+m+mi}{0}\PY{p}{,}\PY{l+m+mi}{0}\PY{p}{]}\PY{p}{,}\PY{p}{[}\PY{o}{\PYZhy{}}\PY{l+m+mi}{1}\PY{p}{,}\PY{o}{\PYZhy{}}\PY{l+m+mi}{2}\PY{p}{,}\PY{o}{\PYZhy{}}\PY{l+m+mi}{1}\PY{p}{]}\PY{p}{]}
        \PY{n}{g\PYZus{}x} \PY{o}{=} \PY{n}{scipy}\PY{o}{.}\PY{n}{signal}\PY{o}{.}\PY{n}{convolve2d}\PY{p}{(}\PY{n}{gauss\PYZus{}img}\PY{p}{,}\PY{n}{h\PYZus{}sobel\PYZus{}x}\PY{p}{)}
        \PY{n}{g\PYZus{}y} \PY{o}{=} \PY{n}{scipy}\PY{o}{.}\PY{n}{signal}\PY{o}{.}\PY{n}{convolve2d}\PY{p}{(}\PY{n}{gauss\PYZus{}img}\PY{p}{,}\PY{n}{h\PYZus{}sobel\PYZus{}y}\PY{p}{)}
        
        \PY{c+c1}{\PYZsh{} \PYZsh{} Calculate the gradient magnitude}
        \PY{n}{g\PYZus{}squared} \PY{o}{=} \PY{n}{np}\PY{o}{.}\PY{n}{add}\PY{p}{(}\PY{n}{np}\PY{o}{.}\PY{n}{square}\PY{p}{(}\PY{n}{g\PYZus{}x}\PY{p}{)}\PY{p}{,}\PY{n}{np}\PY{o}{.}\PY{n}{square}\PY{p}{(}\PY{n}{g\PYZus{}y}\PY{p}{)}\PY{p}{)}
        \PY{n}{sobel\PYZus{}mag} \PY{o}{=} \PY{n}{np}\PY{o}{.}\PY{n}{sqrt}\PY{p}{(}\PY{n}{g\PYZus{}squared}\PY{p}{)}
        
        
        \PY{c+c1}{\PYZsh{} Display the magnitude image (provided)}
        \PY{n}{plt}\PY{o}{.}\PY{n}{imshow}\PY{p}{(}\PY{n}{sobel\PYZus{}mag}\PY{p}{,} \PY{n}{cmap}\PY{o}{=}\PY{l+s+s1}{\PYZsq{}}\PY{l+s+s1}{gray}\PY{l+s+s1}{\PYZsq{}}\PY{p}{,}\PY{n}{vmin}\PY{o}{=}\PY{l+m+mi}{0}\PY{p}{,}\PY{n}{vmax}\PY{o}{=}\PY{l+m+mi}{100}\PY{p}{)}
        \PY{n}{plt}\PY{o}{.}\PY{n}{gcf}\PY{p}{(}\PY{p}{)}\PY{o}{.}\PY{n}{set\PYZus{}size\PYZus{}inches}\PY{p}{(}\PY{l+m+mi}{10}\PY{p}{,} \PY{l+m+mi}{8}\PY{p}{)}
\end{Verbatim}


    \begin{center}
    \adjustimage{max size={0.9\linewidth}{0.9\paperheight}}{output_17_0.png}
    \end{center}
    { \hspace*{\fill} \\}
    
    \subsubsection{\texorpdfstring{2.4 Perform Gaussian smoothing
(\(\sigma\) = 7 pixels) and evaluate the computational time for Gaussian
smoothing. After that, perform Sobel filtering. (9
points)}{2.4 Perform Gaussian smoothing (\textbackslash{}sigma = 7 pixels) and evaluate the computational time for Gaussian smoothing. After that, perform Sobel filtering. (9 points)}}\label{perform-gaussian-smoothing-sigma-7-pixels-and-evaluate-the-computational-time-for-gaussian-smoothing.-after-that-perform-sobel-filtering.-9-points}

    \begin{Verbatim}[commandchars=\\\{\}]
{\color{incolor}In [{\color{incolor}9}]:} \PY{k+kn}{from} \PY{n+nn}{time} \PY{k}{import} \PY{n}{perf\PYZus{}counter}
        \PY{c+c1}{\PYZsh{} Construct the Gaussian filter}
        \PY{n}{gauss\PYZus{}filter} \PY{o}{=} \PY{n}{gaussian\PYZus{}filter\PYZus{}2d}\PY{p}{(}\PY{l+m+mi}{7}\PY{p}{)}
        
        \PY{c+c1}{\PYZsh{} Perform Gaussian smoothing and count time}
        \PY{n}{start} \PY{o}{=} \PY{n}{perf\PYZus{}counter}\PY{p}{(}\PY{p}{)} 
        \PY{n}{gauss\PYZus{}img} \PY{o}{=} \PY{n}{scipy}\PY{o}{.}\PY{n}{signal}\PY{o}{.}\PY{n}{convolve2d}\PY{p}{(}\PY{n}{image\PYZus{}noisy}\PY{p}{,} \PY{n}{gauss\PYZus{}filter}\PY{p}{)}
        \PY{n}{stop} \PY{o}{=} \PY{n}{perf\PYZus{}counter}\PY{p}{(}\PY{p}{)}
        \PY{n}{comp\PYZus{}time} \PY{o}{=} \PY{n}{stop} \PY{o}{\PYZhy{}} \PY{n}{start}
        
        \PY{c+c1}{\PYZsh{} Sobel filtering}
        \PY{n}{h\PYZus{}sobel\PYZus{}x} \PY{o}{=} \PY{p}{[}\PY{p}{[}\PY{l+m+mi}{1}\PY{p}{,}\PY{l+m+mi}{0}\PY{p}{,}\PY{o}{\PYZhy{}}\PY{l+m+mi}{1}\PY{p}{]}\PY{p}{,}\PY{p}{[}\PY{l+m+mi}{2}\PY{p}{,}\PY{l+m+mi}{0}\PY{p}{,}\PY{o}{\PYZhy{}}\PY{l+m+mi}{2}\PY{p}{]}\PY{p}{,}\PY{p}{[}\PY{l+m+mi}{1}\PY{p}{,}\PY{l+m+mi}{0}\PY{p}{,}\PY{o}{\PYZhy{}}\PY{l+m+mi}{1}\PY{p}{]}\PY{p}{]}
        \PY{n}{h\PYZus{}sobel\PYZus{}y} \PY{o}{=} \PY{p}{[}\PY{p}{[}\PY{l+m+mi}{1}\PY{p}{,}\PY{l+m+mi}{2}\PY{p}{,}\PY{l+m+mi}{1}\PY{p}{]}\PY{p}{,}\PY{p}{[}\PY{l+m+mi}{0}\PY{p}{,}\PY{l+m+mi}{0}\PY{p}{,}\PY{l+m+mi}{0}\PY{p}{]}\PY{p}{,}\PY{p}{[}\PY{o}{\PYZhy{}}\PY{l+m+mi}{1}\PY{p}{,}\PY{o}{\PYZhy{}}\PY{l+m+mi}{2}\PY{p}{,}\PY{o}{\PYZhy{}}\PY{l+m+mi}{1}\PY{p}{]}\PY{p}{]}
        \PY{n}{g\PYZus{}x} \PY{o}{=} \PY{n}{scipy}\PY{o}{.}\PY{n}{signal}\PY{o}{.}\PY{n}{convolve2d}\PY{p}{(}\PY{n}{gauss\PYZus{}img}\PY{p}{,}\PY{n}{h\PYZus{}sobel\PYZus{}x}\PY{p}{)}
        \PY{n}{g\PYZus{}y} \PY{o}{=} \PY{n}{scipy}\PY{o}{.}\PY{n}{signal}\PY{o}{.}\PY{n}{convolve2d}\PY{p}{(}\PY{n}{gauss\PYZus{}img}\PY{p}{,}\PY{n}{h\PYZus{}sobel\PYZus{}y}\PY{p}{)}
        
        \PY{c+c1}{\PYZsh{} Calculate the gradient magnitude}
        \PY{n}{g\PYZus{}squared} \PY{o}{=} \PY{n}{np}\PY{o}{.}\PY{n}{add}\PY{p}{(}\PY{n}{np}\PY{o}{.}\PY{n}{square}\PY{p}{(}\PY{n}{g\PYZus{}x}\PY{p}{)}\PY{p}{,}\PY{n}{np}\PY{o}{.}\PY{n}{square}\PY{p}{(}\PY{n}{g\PYZus{}y}\PY{p}{)}\PY{p}{)}
        \PY{n}{sobel\PYZus{}mag} \PY{o}{=} \PY{n}{np}\PY{o}{.}\PY{n}{sqrt}\PY{p}{(}\PY{n}{g\PYZus{}squared}\PY{p}{)}
        
        \PY{c+c1}{\PYZsh{} Display the magnitude image (provided)}
        \PY{n+nb}{print}\PY{p}{(}\PY{n}{comp\PYZus{}time}\PY{p}{)}
        \PY{n}{plt}\PY{o}{.}\PY{n}{imshow}\PY{p}{(}\PY{n}{sobel\PYZus{}mag}\PY{p}{,} \PY{n}{cmap}\PY{o}{=}\PY{l+s+s1}{\PYZsq{}}\PY{l+s+s1}{gray}\PY{l+s+s1}{\PYZsq{}}\PY{p}{,}\PY{n}{vmin}\PY{o}{=}\PY{l+m+mi}{0}\PY{p}{,}\PY{n}{vmax}\PY{o}{=}\PY{l+m+mi}{100}\PY{p}{)}
        \PY{n}{plt}\PY{o}{.}\PY{n}{gcf}\PY{p}{(}\PY{p}{)}\PY{o}{.}\PY{n}{set\PYZus{}size\PYZus{}inches}\PY{p}{(}\PY{l+m+mi}{10}\PY{p}{,} \PY{l+m+mi}{8}\PY{p}{)}
\end{Verbatim}


    \begin{Verbatim}[commandchars=\\\{\}]
41.552064699993934

    \end{Verbatim}

    \begin{center}
    \adjustimage{max size={0.9\linewidth}{0.9\paperheight}}{output_19_1.png}
    \end{center}
    { \hspace*{\fill} \\}
    
    \subsubsection{\texorpdfstring{2.5 Implement a function that generates a
1D Gaussian filter given the parameter \(\sigma\). Generate 1D Gaussian
filters along x-axis and y-axis respectively. (10
points)}{2.5 Implement a function that generates a 1D Gaussian filter given the parameter \textbackslash{}sigma. Generate 1D Gaussian filters along x-axis and y-axis respectively. (10 points)}}\label{implement-a-function-that-generates-a-1d-gaussian-filter-given-the-parameter-sigma.-generate-1d-gaussian-filters-along-x-axis-and-y-axis-respectively.-10-points}

    \begin{Verbatim}[commandchars=\\\{\}]
{\color{incolor}In [{\color{incolor}10}]:} \PY{c+c1}{\PYZsh{} Design the Gaussian filter}
         \PY{k}{def} \PY{n+nf}{gaussian\PYZus{}filter\PYZus{}1d}\PY{p}{(}\PY{n}{sigma}\PY{p}{)}\PY{p}{:}
             \PY{c+c1}{\PYZsh{} sigma: the parameter sigma in the Gaussian kernel (unit: pixel)}
             \PY{c+c1}{\PYZsh{}}
             \PY{c+c1}{\PYZsh{} return: a 1D array for the Gaussian kernel}
             \PY{n}{constant} \PY{o}{=} \PY{l+m+mi}{1}\PY{o}{/}\PY{p}{(}\PY{n}{np}\PY{o}{.}\PY{n}{sqrt}\PY{p}{(}\PY{l+m+mi}{2}\PY{o}{*}\PY{n}{np}\PY{o}{.}\PY{n}{pi}\PY{p}{)}\PY{o}{*}\PY{n}{sigma}\PY{p}{)}
             \PY{c+c1}{\PYZsh{}print(constant)}
             \PY{n}{k} \PY{o}{=} \PY{l+m+mi}{4}
             \PY{n}{sz} \PY{o}{=} \PY{n+nb}{int}\PY{p}{(}\PY{l+m+mi}{2}\PY{o}{*}\PY{n}{sigma} \PY{o}{*} \PY{n}{k}\PY{p}{)}\PY{o}{+}\PY{l+m+mi}{1} \PY{c+c1}{\PYZsh{}kernel size}
             \PY{n}{var} \PY{o}{=} \PY{n}{sigma}\PY{o}{*}\PY{o}{*}\PY{l+m+mi}{2} \PY{c+c1}{\PYZsh{}variance}
             \PY{n}{h} \PY{o}{=} \PY{p}{[}\PY{p}{[}\PY{l+m+mi}{0}\PY{p}{]}\PY{o}{*}\PY{n}{sz}\PY{p}{]}
             \PY{c+c1}{\PYZsh{}print(h)}
             \PY{k}{for} \PY{n}{i} \PY{o+ow}{in} \PY{n+nb}{range}\PY{p}{(}\PY{n}{sz}\PY{p}{)}\PY{p}{:}
                 \PY{n}{x} \PY{o}{=} \PY{n}{i}\PY{o}{\PYZhy{}}\PY{p}{(}\PY{n}{k}\PY{o}{*}\PY{n}{sigma}\PY{p}{)}
                 \PY{n}{h}\PY{p}{[}\PY{l+m+mi}{0}\PY{p}{]}\PY{p}{[}\PY{n}{i}\PY{p}{]}\PY{o}{=}\PY{n}{constant}\PY{o}{*}\PY{n}{np}\PY{o}{.}\PY{n}{exp}\PY{p}{(}\PY{o}{\PYZhy{}}\PY{p}{(}\PY{n}{x}\PY{o}{*}\PY{o}{*}\PY{l+m+mi}{2}\PY{p}{)}\PY{o}{/}\PY{p}{(}\PY{n}{var}\PY{o}{*}\PY{l+m+mi}{2}\PY{p}{)}\PY{p}{)}
             \PY{k}{return} \PY{n}{h}
         
         \PY{c+c1}{\PYZsh{} sigma = 7 pixel (provided)}
         \PY{n}{sigma} \PY{o}{=} \PY{l+m+mi}{7}
         
         \PY{c+c1}{\PYZsh{} The Gaussian filter along x\PYZhy{}axis. Its shape is (1, sz).}
         \PY{n}{h\PYZus{}x} \PY{o}{=} \PY{n}{gaussian\PYZus{}filter\PYZus{}1d}\PY{p}{(}\PY{n}{sigma}\PY{p}{)}
         \PY{c+c1}{\PYZsh{}print(h\PYZus{}x)}
         \PY{c+c1}{\PYZsh{}print(np.shape(h\PYZus{}x))}
         \PY{c+c1}{\PYZsh{} The Gaussian filter along y\PYZhy{}axis. Its shape is (sz, 1).}
         \PY{n}{h\PYZus{}y} \PY{o}{=} \PY{n}{np}\PY{o}{.}\PY{n}{transpose}\PY{p}{(}\PY{n}{gaussian\PYZus{}filter\PYZus{}1d}\PY{p}{(}\PY{n}{sigma}\PY{p}{)}\PY{p}{)}
         \PY{c+c1}{\PYZsh{}print(h\PYZus{}y)}
         \PY{c+c1}{\PYZsh{}print(np.shape(h\PYZus{}y))}
         \PY{c+c1}{\PYZsh{} Visualise the filters (provided)}
         \PY{n}{plt}\PY{o}{.}\PY{n}{subplot}\PY{p}{(}\PY{l+m+mi}{1}\PY{p}{,} \PY{l+m+mi}{2}\PY{p}{,} \PY{l+m+mi}{1}\PY{p}{)}
         \PY{n}{plt}\PY{o}{.}\PY{n}{imshow}\PY{p}{(}\PY{n}{h\PYZus{}x}\PY{p}{)}
         \PY{n}{plt}\PY{o}{.}\PY{n}{subplot}\PY{p}{(}\PY{l+m+mi}{1}\PY{p}{,} \PY{l+m+mi}{2}\PY{p}{,} \PY{l+m+mi}{2}\PY{p}{)}
         \PY{n}{plt}\PY{o}{.}\PY{n}{imshow}\PY{p}{(}\PY{n}{h\PYZus{}y}\PY{p}{)}
\end{Verbatim}


\begin{Verbatim}[commandchars=\\\{\}]
{\color{outcolor}Out[{\color{outcolor}10}]:} <matplotlib.image.AxesImage at 0x7f0b4ff9e1d0>
\end{Verbatim}
            
    \begin{center}
    \adjustimage{max size={0.9\linewidth}{0.9\paperheight}}{output_21_1.png}
    \end{center}
    { \hspace*{\fill} \\}
    
    \subsubsection{2.6 Perform Gaussian smoothing (sigma = 7 pixels) using
two separable filters and evaluate the computational time for separable
Gaussian filtering. After that, perform Sobel filtering and show
results. (9
points)}\label{perform-gaussian-smoothing-sigma-7-pixels-using-two-separable-filters-and-evaluate-the-computational-time-for-separable-gaussian-filtering.-after-that-perform-sobel-filtering-and-show-results.-9-points}

    \begin{Verbatim}[commandchars=\\\{\}]
{\color{incolor}In [{\color{incolor}11}]:} \PY{c+c1}{\PYZsh{} Perform separable Gaussian smoothing and count time}
         \PY{n}{h\PYZus{}x} \PY{o}{=} \PY{n}{gaussian\PYZus{}filter\PYZus{}1d}\PY{p}{(}\PY{l+m+mi}{7}\PY{p}{)}
         \PY{n}{h\PYZus{}y} \PY{o}{=} \PY{n}{np}\PY{o}{.}\PY{n}{transpose}\PY{p}{(}\PY{n}{gaussian\PYZus{}filter\PYZus{}1d}\PY{p}{(}\PY{l+m+mi}{7}\PY{p}{)}\PY{p}{)}
         \PY{n}{start} \PY{o}{=} \PY{n}{perf\PYZus{}counter}\PY{p}{(}\PY{p}{)} 
         \PY{n}{gauss\PYZus{}img} \PY{o}{=} \PY{n}{scipy}\PY{o}{.}\PY{n}{signal}\PY{o}{.}\PY{n}{convolve2d}\PY{p}{(}\PY{n}{image\PYZus{}noisy}\PY{p}{,} \PY{n}{h\PYZus{}x}\PY{p}{)}
         \PY{n}{gauss\PYZus{}img} \PY{o}{=} \PY{n}{scipy}\PY{o}{.}\PY{n}{signal}\PY{o}{.}\PY{n}{convolve2d}\PY{p}{(}\PY{n}{gauss\PYZus{}img}\PY{p}{,} \PY{n}{h\PYZus{}y}\PY{p}{)}
         \PY{n}{stop} \PY{o}{=} \PY{n}{perf\PYZus{}counter}\PY{p}{(}\PY{p}{)}
         \PY{n}{comp\PYZus{}time} \PY{o}{=} \PY{n}{stop} \PY{o}{\PYZhy{}} \PY{n}{start}
         
         \PY{c+c1}{\PYZsh{} Sobel filtering}
         \PY{n}{h\PYZus{}sobel\PYZus{}x} \PY{o}{=} \PY{p}{[}\PY{p}{[}\PY{l+m+mi}{1}\PY{p}{,}\PY{l+m+mi}{0}\PY{p}{,}\PY{o}{\PYZhy{}}\PY{l+m+mi}{1}\PY{p}{]}\PY{p}{,}\PY{p}{[}\PY{l+m+mi}{2}\PY{p}{,}\PY{l+m+mi}{0}\PY{p}{,}\PY{o}{\PYZhy{}}\PY{l+m+mi}{2}\PY{p}{]}\PY{p}{,}\PY{p}{[}\PY{l+m+mi}{1}\PY{p}{,}\PY{l+m+mi}{0}\PY{p}{,}\PY{o}{\PYZhy{}}\PY{l+m+mi}{1}\PY{p}{]}\PY{p}{]}
         \PY{n}{h\PYZus{}sobel\PYZus{}y} \PY{o}{=} \PY{p}{[}\PY{p}{[}\PY{l+m+mi}{1}\PY{p}{,}\PY{l+m+mi}{2}\PY{p}{,}\PY{l+m+mi}{1}\PY{p}{]}\PY{p}{,}\PY{p}{[}\PY{l+m+mi}{0}\PY{p}{,}\PY{l+m+mi}{0}\PY{p}{,}\PY{l+m+mi}{0}\PY{p}{]}\PY{p}{,}\PY{p}{[}\PY{o}{\PYZhy{}}\PY{l+m+mi}{1}\PY{p}{,}\PY{o}{\PYZhy{}}\PY{l+m+mi}{2}\PY{p}{,}\PY{o}{\PYZhy{}}\PY{l+m+mi}{1}\PY{p}{]}\PY{p}{]}
         \PY{n}{g\PYZus{}x} \PY{o}{=} \PY{n}{scipy}\PY{o}{.}\PY{n}{signal}\PY{o}{.}\PY{n}{convolve2d}\PY{p}{(}\PY{n}{gauss\PYZus{}img}\PY{p}{,}\PY{n}{h\PYZus{}sobel\PYZus{}x}\PY{p}{)}
         \PY{n}{g\PYZus{}y} \PY{o}{=} \PY{n}{scipy}\PY{o}{.}\PY{n}{signal}\PY{o}{.}\PY{n}{convolve2d}\PY{p}{(}\PY{n}{gauss\PYZus{}img}\PY{p}{,}\PY{n}{h\PYZus{}sobel\PYZus{}y}\PY{p}{)}
         
         \PY{c+c1}{\PYZsh{} Calculate the gradient magnitude}
         \PY{n}{g\PYZus{}squared} \PY{o}{=} \PY{n}{np}\PY{o}{.}\PY{n}{add}\PY{p}{(}\PY{n}{np}\PY{o}{.}\PY{n}{square}\PY{p}{(}\PY{n}{g\PYZus{}x}\PY{p}{)}\PY{p}{,}\PY{n}{np}\PY{o}{.}\PY{n}{square}\PY{p}{(}\PY{n}{g\PYZus{}y}\PY{p}{)}\PY{p}{)}
         \PY{n}{sobel\PYZus{}mag} \PY{o}{=} \PY{n}{np}\PY{o}{.}\PY{n}{sqrt}\PY{p}{(}\PY{n}{g\PYZus{}squared}\PY{p}{)}
         
         \PY{c+c1}{\PYZsh{} Display the magnitude image (provided)}
         \PY{n+nb}{print}\PY{p}{(}\PY{n}{comp\PYZus{}time}\PY{p}{)}
         \PY{n}{plt}\PY{o}{.}\PY{n}{imshow}\PY{p}{(}\PY{n}{sobel\PYZus{}mag}\PY{p}{,} \PY{n}{cmap}\PY{o}{=}\PY{l+s+s1}{\PYZsq{}}\PY{l+s+s1}{gray}\PY{l+s+s1}{\PYZsq{}}\PY{p}{,}\PY{n}{vmin}\PY{o}{=}\PY{l+m+mi}{0}\PY{p}{,}\PY{n}{vmax}\PY{o}{=}\PY{l+m+mi}{100}\PY{p}{)}
         \PY{n}{plt}\PY{o}{.}\PY{n}{gcf}\PY{p}{(}\PY{p}{)}\PY{o}{.}\PY{n}{set\PYZus{}size\PYZus{}inches}\PY{p}{(}\PY{l+m+mi}{10}\PY{p}{,} \PY{l+m+mi}{8}\PY{p}{)}
\end{Verbatim}


    \begin{Verbatim}[commandchars=\\\{\}]
3.3382424999726936

    \end{Verbatim}

    \begin{center}
    \adjustimage{max size={0.9\linewidth}{0.9\paperheight}}{output_23_1.png}
    \end{center}
    { \hspace*{\fill} \\}
    
    \subsubsection{2.7 Comment on the Gaussian + Sobel filtering results and
the computational time. (10
points)}\label{comment-on-the-gaussian-sobel-filtering-results-and-the-computational-time.-10-points}

    Gaussian filtering supresses noises which creates a smoothing effect.
Sobel Filter does further smoothing and also provides edge detection.
The computational time is reduced signficantly(10x) by using 2 separable
filters instead of 1 combine one. This is because

    \subsection{3. Challenge: Implement the 3x3 moving average filter using
Pytorch. (15
points)}\label{challenge-implement-the-3x3-moving-average-filter-using-pytorch.-15-points}

\href{https://pytorch.org/}{Pytorch} is a machine learning framework
that supports filtering and convolution.

The \href{https://pytorch.org/docs/stable/nn.html\#conv2d}{Conv2D}
operator takes an input array of dimension NxC1xXxY, applies the filter
and outputs an array of dimension NxC2xXxY. Here, since we only have one
image with one colour channel, we will set N=1, C1=1 and C2=1. You can
read the documentation of Conv2D for more detail.

    \begin{Verbatim}[commandchars=\\\{\}]
{\color{incolor}In [{\color{incolor}12}]:} \PY{c+c1}{\PYZsh{} Import libaries (provided)}
         \PY{k+kn}{import} \PY{n+nn}{torch}
\end{Verbatim}


    \subsubsection{3.1 Expand the dimension of the noisy image into 1x1xXxY
and convert it to a Pytorch tensor. (5
points)}\label{expand-the-dimension-of-the-noisy-image-into-1x1xxxy-and-convert-it-to-a-pytorch-tensor.-5-points}

    \begin{Verbatim}[commandchars=\\\{\}]
{\color{incolor}In [{\color{incolor}13}]:} \PY{c+c1}{\PYZsh{} Expand the dimension of the numpy array}
         \PY{n}{y} \PY{o}{=} \PY{n}{np}\PY{o}{.}\PY{n}{expand\PYZus{}dims}\PY{p}{(}\PY{n}{np}\PY{o}{.}\PY{n}{expand\PYZus{}dims}\PY{p}{(}\PY{n}{image\PYZus{}noisy}\PY{p}{,} \PY{n}{axis}\PY{o}{=}\PY{l+m+mi}{0}\PY{p}{)}\PY{p}{,}\PY{n}{axis}\PY{o}{=}\PY{l+m+mi}{0}\PY{p}{)}
         
         \PY{n+nb}{print}\PY{p}{(}\PY{n}{y}\PY{o}{.}\PY{n}{shape}\PY{p}{)}
         \PY{c+c1}{\PYZsh{} Convert to a Pytorch tensor using torch.from\PYZus{}numpy}
         \PY{n}{t} \PY{o}{=} \PY{n}{torch}\PY{o}{.}\PY{n}{from\PYZus{}numpy}\PY{p}{(}\PY{n}{y}\PY{p}{)}\PY{o}{.}\PY{n}{float}\PY{p}{(}\PY{p}{)}
         \PY{n+nb}{print}\PY{p}{(}\PY{n}{t}\PY{o}{.}\PY{n}{size}\PY{p}{(}\PY{p}{)}\PY{p}{)}
\end{Verbatim}


    \begin{Verbatim}[commandchars=\\\{\}]
(1, 1, 1800, 2400)
torch.Size([1, 1, 1800, 2400])

    \end{Verbatim}

    \subsubsection{3.2 Create a Pytorch Conv2D filter, set its kernel to be
a 3x3 moving averaging filter. (5
points)}\label{create-a-pytorch-conv2d-filter-set-its-kernel-to-be-a-3x3-moving-averaging-filter.-5-points}

    \begin{Verbatim}[commandchars=\\\{\}]
{\color{incolor}In [{\color{incolor}24}]:} \PY{c+c1}{\PYZsh{} Create the Conv2D filter (provided)}
         \PY{n}{conv} \PY{o}{=} \PY{n}{torch}\PY{o}{.}\PY{n}{nn}\PY{o}{.}\PY{n}{Conv2d}\PY{p}{(}\PY{n}{in\PYZus{}channels}\PY{o}{=}\PY{l+m+mi}{1}\PY{p}{,} \PY{n}{out\PYZus{}channels}\PY{o}{=}\PY{l+m+mi}{1}\PY{p}{,} \PY{n}{kernel\PYZus{}size}\PY{o}{=}\PY{l+m+mi}{3}\PY{p}{,} \PY{n}{padding}\PY{o}{=}\PY{l+m+mi}{1}\PY{p}{,} \PY{n}{bias}\PY{o}{=}\PY{k+kc}{False}\PY{p}{)}
         
         \PY{c+c1}{\PYZsh{} Set the kernel weight}
         \PY{n}{mov\PYZus{}avg} \PY{o}{=} \PY{n}{torch}\PY{o}{.}\PY{n}{Tensor}\PY{p}{(}\PY{p}{[}\PY{p}{[}\PY{l+m+mi}{1}\PY{o}{/}\PY{l+m+mi}{9} \PY{p}{,}\PY{l+m+mi}{1}\PY{o}{/}\PY{l+m+mi}{9}\PY{p}{,} \PY{l+m+mi}{1}\PY{o}{/}\PY{l+m+mi}{9}\PY{p}{]}\PY{p}{,}\PY{p}{[}\PY{l+m+mi}{1}\PY{o}{/}\PY{l+m+mi}{9}\PY{p}{,} \PY{l+m+mi}{1}\PY{o}{/}\PY{l+m+mi}{9} \PY{p}{,}\PY{l+m+mi}{1}\PY{o}{/}\PY{l+m+mi}{9}\PY{p}{]}\PY{p}{,} \PY{p}{[}\PY{l+m+mi}{1}\PY{o}{/}\PY{l+m+mi}{9}\PY{p}{,} \PY{l+m+mi}{1}\PY{o}{/}\PY{l+m+mi}{9} \PY{p}{,}\PY{l+m+mi}{1}\PY{o}{/}\PY{l+m+mi}{9}\PY{p}{]}\PY{p}{]}\PY{p}{)}\PY{o}{.}\PY{n}{expand}\PY{p}{(}\PY{n}{conv}\PY{o}{.}\PY{n}{weight}\PY{o}{.}\PY{n}{size}\PY{p}{(}\PY{p}{)}\PY{p}{)}
         \PY{n}{kernel\PYZus{}weights} \PY{o}{=} \PY{n}{torch}\PY{o}{.}\PY{n}{nn}\PY{o}{.}\PY{n}{Parameter}\PY{p}{(}\PY{n}{mov\PYZus{}avg}\PY{p}{)}
         \PY{n}{conv}\PY{o}{.}\PY{n}{weight} \PY{o}{=} \PY{n}{kernel\PYZus{}weights}
         \PY{n+nb}{print}\PY{p}{(}\PY{n}{conv}\PY{o}{.}\PY{n}{weight} \PY{p}{)}
\end{Verbatim}


    \begin{Verbatim}[commandchars=\\\{\}]
Parameter containing:
tensor([[[[0.1111, 0.1111, 0.1111],
          [0.1111, 0.1111, 0.1111],
          [0.1111, 0.1111, 0.1111]]]], requires\_grad=True)

    \end{Verbatim}

    \subsubsection{3.3 Apply the filter to the noisy image tensor and
display the output image. (5
points)}\label{apply-the-filter-to-the-noisy-image-tensor-and-display-the-output-image.-5-points}

    \begin{Verbatim}[commandchars=\\\{\}]
{\color{incolor}In [{\color{incolor}25}]:} \PY{n}{conv}\PY{p}{(}\PY{n}{t}\PY{p}{)}
         
         \PY{c+c1}{\PYZsh{} Display the filtering result (provided)}
         \PY{n}{plt}\PY{o}{.}\PY{n}{imshow}\PY{p}{(}\PY{n}{image\PYZus{}filtered}\PY{p}{,} \PY{n}{cmap}\PY{o}{=}\PY{l+s+s1}{\PYZsq{}}\PY{l+s+s1}{gray}\PY{l+s+s1}{\PYZsq{}}\PY{p}{)}
         \PY{n}{plt}\PY{o}{.}\PY{n}{gcf}\PY{p}{(}\PY{p}{)}\PY{o}{.}\PY{n}{set\PYZus{}size\PYZus{}inches}\PY{p}{(}\PY{l+m+mi}{10}\PY{p}{,} \PY{l+m+mi}{8}\PY{p}{)}
\end{Verbatim}


    \begin{center}
    \adjustimage{max size={0.9\linewidth}{0.9\paperheight}}{output_33_0.png}
    \end{center}
    { \hspace*{\fill} \\}
    
    \subsection{4. Survey: How long does it take you to complete the
coursework?}\label{survey-how-long-does-it-take-you-to-complete-the-coursework}

    \begin{Verbatim}[commandchars=\\\{\}]
{\color{incolor}In [{\color{incolor} }]:} \PY{n}{Six} \PY{n}{Hours}
\end{Verbatim}



    % Add a bibliography block to the postdoc
    
    
    
    \end{document}
